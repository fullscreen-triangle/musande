\documentclass[12pt,a4paper]{article}
\usepackage[utf8]{inputenc}
\usepackage[T1]{fontenc}
\usepackage{amsmath,amssymb,amsfonts}
\usepackage{graphicx}
\usepackage{float}
\usepackage{tikz}
\usepackage{pgfplots}
\usepackage{booktabs}
\usepackage{array}
\usepackage{siunitx}
\usepackage{physics}
\usepackage{cite}
\usepackage{url}
\usepackage{hyperref}
\usepackage{geometry}
\usepackage{fancyhdr}
\usepackage{subcaption}

\geometry{margin=1in}
\pagestyle{fancy}
\fancyhf{}
\rhead{\thepage}
\lhead{Underwater Supersonic Ballistics}

\title{\textbf{Achieving Mach 1.7 in Water: A Novel Micro-Scale Ballistics Experiment for Supersonic Fluid Dynamics Investigation}}

\author{
Kundai Farai Sachikonye\\
\textit{Independent Research}\\
\textit{Theoretical Physics and Advanced Engineering}\\
\textit{Buhera, Zimbabwe}\\
\texttt{kundai.sachikonye@wzw.tum.de}
}

\date{\today}

\begin{document}

\maketitle

\begin{abstract}
We present a novel experimental approach to achieve supersonic velocities in aqueous media through a sequential three-stage projectile system. The proposed experiment aims to reach Mach 1.7 ($v = 2500$ m/s) in water using engineered fluid mixtures and progressive heating systems. Our methodology employs automated ballistics firing sequential projectiles of decreasing mass (50g $\rightarrow$ 15g $\rightarrow$ 0.8g) through a 1-meter instrumented track containing specially formulated low-surface-tension fluid mixtures. Theoretical analysis indicates achievable supersonic transitions with observable plasma formation, Čerenkov radiation, and acoustic shock signatures. Energy requirements are calculated at $\sim$3 kJ total, making this approach feasible with existing laboratory equipment. Expected phenomena include localized plasma channels reaching $>100,000$ K, pressure waves exceeding 5 GPa, and electromagnetic signatures suitable for comprehensive supersonic fluid dynamics analysis. This work represents the first practical framework for controlled underwater supersonic ballistics experimentation.

\textbf{Keywords:} supersonic fluid dynamics, underwater ballistics, plasma formation, Čerenkov radiation, shock wave propagation, cavitation dynamics
\end{abstract}

\section{Introduction}

The achievement of supersonic velocities in liquid media represents one of the most challenging frontiers in experimental fluid dynamics. While supersonic flight in atmospheric conditions has been thoroughly investigated since the 1940s \cite{anderson2003fundamentals, bertin2013hypersonic}, supersonic motion through dense liquid media remains largely unexplored due to the extreme energy requirements and complex multi-phase phenomena involved.

Water conducts sound at approximately 1,480 m/s under standard conditions \cite{kinsler2000fundamentals}, significantly faster than the 343 m/s propagation velocity in air \cite{pierce2019acoustics}. Consequently, achieving supersonic speeds underwater (Mach $> 1$) requires velocities exceeding 1,480 m/s, while our target of Mach 1.7 necessitates reaching 2,500 m/s—a feat that has not been achieved under controlled laboratory conditions.

\subsection{Theoretical Background}

The fundamental challenge of underwater supersonic motion stems from the density differential between water and air. Water exhibits a density of 1000 kg/m³ compared to air's 1.225 kg/m³ at standard conditions \cite{white2016fluid}, creating resistance forces approximately 800 times greater than atmospheric supersonic flight. This necessitates proportionally higher energy inputs and results in complex multi-phase transitions including cavitation \cite{brennen2014cavitation}, plasma formation \cite{fridman2008plasma}, and shock wave propagation \cite{glass1974shock}.

Previous investigations of high-velocity projectile motion in water have been limited to subsonic regimes \cite{hirt1974volume, shi2010numerical}. Cavitation studies have examined bubble dynamics at lower velocities \cite{plesset1977bubble}, while plasma formation in liquids has been primarily investigated through electrical discharge methods rather than ballistic approaches \cite{bruggeman2002plasma}.

\subsection{Research Objectives}

This investigation aims to:
\begin{enumerate}
\item Develop a practical experimental framework for achieving supersonic velocities in aqueous media
\item Characterize the multi-phase phenomena occurring during underwater supersonic transitions
\item Quantify energy requirements and optimization strategies for sustained supersonic motion
\item Establish measurement protocols for plasma formation, shock wave propagation, and electromagnetic signatures
\end{enumerate}

\section{Theoretical Framework}

\subsection{Energy Requirements Analysis}

The total energy required for supersonic underwater motion can be decomposed into four primary components:

\begin{equation}
E_{\text{total}} = E_{\text{vap}} + E_{\text{ion}} + E_{\text{kin}} + E_{\text{shock}}
\label{eq:total_energy}
\end{equation}

where:
\begin{align}
E_{\text{vap}} &= h_{\text{fg}} \cdot m_{\text{fluid}} \approx 2.3 \text{ MJ/kg} \label{eq:vaporization}\\
E_{\text{ion}} &= \sum_{i} I_i \cdot N_i \approx 13.6 \text{ eV/atom} \label{eq:ionization}\\
E_{\text{kin}} &= \frac{1}{2}mv^2 \approx 2.5 \text{ kJ for 0.8g at 2500 m/s} \label{eq:kinetic}\\
E_{\text{shock}} &= \eta_{\text{shock}} \cdot E_{\text{kin}} \approx 0.6 \cdot E_{\text{kin}} \label{eq:shock}
\end{align}

The vaporization energy $E_{\text{vap}}$ represents the latent heat required for liquid-to-vapor phase transitions \cite{cengel2019thermodynamics}. Ionization energy $E_{\text{ion}}$ accounts for plasma formation when local temperatures exceed 10,000 K \cite{chen2016introduction}. The shock energy $E_{\text{shock}}$ typically constitutes 60\% of kinetic energy in supersonic fluid motion \cite{sedov1993similarity}.

\subsection{Drag Force Evolution in Multi-Phase Flow}

The drag force experienced by a projectile transitioning through engineered fluid media exhibits complex dependencies on local thermodynamic conditions:

\begin{equation}
F_d = \frac{1}{2}\rho(T,P) \cdot C_d(\text{Re},\text{Ma}) \cdot A(t) \cdot v^2 \cdot \phi(x,t)
\label{eq:drag_force}
\end{equation}

where $\rho(T,P)$ represents the temperature and pressure-dependent fluid density, $C_d(\text{Re},\text{Ma})$ is the Reynolds and Mach number-dependent drag coefficient \cite{schlichting2016boundary}, $A(t)$ denotes the time-dependent projectile cross-sectional area, and $\phi(x,t)$ represents the phase transition factor accounting for liquid-vapor-plasma transitions.

The Reynolds number for high-velocity underwater motion is given by:
\begin{equation}
\text{Re} = \frac{\rho v L}{\mu}
\label{eq:reynolds}
\end{equation}

where $L$ is the characteristic length scale and $\mu$ is the dynamic viscosity. For our experimental conditions, $\text{Re} \sim 10^8$, indicating highly turbulent flow regimes \cite{pope2000turbulent}.

\subsection{Plasma Channel Dynamics}

The plasma channel formation following supersonic projectile motion can be modeled using the electron density evolution equation:

\begin{equation}
n_e(x,t) = n_0 \exp\left(-\frac{(x-vt)^2}{2\sigma^2}\right) \cdot \exp\left(-\frac{t}{\tau_p}\right)
\label{eq:plasma_density}
\end{equation}

where $n_0$ represents the peak electron density ($\sim 10^{20}$ electrons/m³), $\sigma$ characterizes the spatial distribution width, and $\tau_p$ is the plasma recombination timescale \cite{lieberman2005principles}.

The plasma recombination time can be estimated from:
\begin{equation}
\tau_p = \frac{1}{\alpha_r n_e}
\label{eq:recombination_time}
\end{equation}

where $\alpha_r$ is the recombination coefficient for water plasma \cite{fridman2008plasma}.

\subsection{Maximum Achievable Velocity}

Physical constraints impose fundamental limits on achievable velocities in dense media. The maximum velocity can be derived from energy conservation principles:

\begin{equation}
v_{\max} = c_w\sqrt{1 + \frac{2E_{\text{input}}}{\rho V c_w^2}}
\label{eq:max_velocity}
\end{equation}

where $c_w = 1480$ m/s is the sound speed in water, $E_{\text{input}}$ represents the total input energy, $\rho$ is the local fluid density, and $V$ is the affected volume \cite{landau1987fluid}.

For our target velocity of 2,500 m/s (Mach 1.7), the required energy input is approximately 3 kJ, which is achievable with conventional ballistic systems.

\section{Experimental Design}

\subsection{Sequential Projectile System}

Our experimental approach employs an automated rifle system firing three projectiles in rapid succession through a 1-meter instrumented track. This sequential approach overcomes the energy limitations of single-projectile systems by progressively conditioning the fluid medium.

\subsubsection{Stage 1: Path Preparation}
\textbf{Projectile specifications:} 50g mass, target velocity 800-1000 m/s
\begin{itemize}
\item Creates initial cavitation tunnel through momentum transfer
\item Displaces bulk fluid volume, reducing effective density
\item Forms low-pressure corridor for subsequent projectiles
\item Initiates acoustic pre-conditioning of the medium
\end{itemize}

\subsubsection{Stage 2: Channel Enhancement}
\textbf{Projectile specifications:} 15g mass, target velocity 1200-1400 m/s
\begin{itemize}
\item Expands and stabilizes the cavitation tunnel
\item Initiates ionization through high-current electrical discharge
\item Creates preliminary plasma channel conditions
\item Establishes thermal gradients for final stage optimization
\end{itemize}

\subsubsection{Stage 3: Supersonic Achievement}
\textbf{Projectile specifications:} 0.8g mass, target velocity 2000-2500 m/s
\begin{itemize}
\item Travels through prepared, low-resistance channel
\item Achieves sustained Mach 1.7 for measurable duration
\item Generates observable supersonic phenomena
\item Enables comprehensive diagnostic measurements
\end{itemize}

\subsection{Engineered Fluid System}

The base fluid mixture is formulated to optimize supersonic transition characteristics:

\begin{table}[H]
\centering
\caption{Engineered Fluid Composition and Properties}
\label{tab:fluid_composition}
\begin{tabular}{lcccc}
\toprule
Component & Percentage & Density (kg/m³) & Surface Tension (mN/m) & Conductivity (S/m) \\
\midrule
Water & 60\% & 1000 & 72.0 & $5.5 \times 10^{-6}$ \\
Acetone & 20\% & 784 & 23.7 & $1.3 \times 10^{-7}$ \\
Methanol & 15\% & 792 & 22.6 & $1.5 \times 10^{-7}$ \\
Liquid N$_2$ & 5\% & 808 & 8.9 & - \\
\midrule
\textbf{Mixture} & \textbf{100\%} & \textbf{932} & \textbf{25.3} & \textbf{Enhanced} \\
\bottomrule
\end{tabular}
\end{table}

The acetone and methanol components reduce surface tension by $\sim$65\%, facilitating cavitation inception and bubble dynamics \cite{blake1987cavitation}. Liquid nitrogen provides localized cooling and density modification, while conductivity enhancers enable electrical discharge effects.

\subsection{Progressive Heating System}

The track incorporates ten discrete heating zones, each 10 cm in length, with temperatures progressing from 300 K to 525 K:

\begin{equation}
T(x) = T_0 + \Delta T \cdot \frac{x}{L_{\text{track}}}
\label{eq:temperature_profile}
\end{equation}

where $T_0 = 300$ K, $\Delta T = 225$ K, and $L_{\text{track}} = 1$ m.

This thermal gradient serves multiple purposes:
\begin{itemize}
\item Reduces local fluid density: $\rho(T) = \rho_0[1 - \beta(T - T_0)]$ \cite{cengel2019thermodynamics}
\item Modifies viscosity: $\mu(T) = \mu_0 \exp(-E_a/RT)$ \cite{bird2006transport}
\item Facilitates ionization at elevated temperatures
\item Creates optimal conditions for plasma formation
\end{itemize}

\section{Expected Phenomena and Measurements}

\subsection{Acoustic Signatures: The Underwater Sonic Boom}

The primary acoustic signature will be an intense pressure wave with the following characteristics:

\begin{align}
\text{Duration:} &\quad 50-100 \text{ nanoseconds} \\
\text{Peak Pressure:} &\quad \Delta P \sim 5 \text{ GPa} \\
\text{Frequency Content:} &\quad f \sim 10^{7}-10^{8} \text{ Hz}
\end{align}

The pressure amplitude can be estimated using the Rankine-Hugoniot relations for shock waves \cite{zel1967physics}:

\begin{equation}
\Delta P = \rho_0 c_0 \Delta u = \rho_0 c_0 (v - c_0)
\label{eq:shock_pressure}
\end{equation}

For $v = 2500$ m/s in water: $\Delta P = 1000 \times 1480 \times (2500 - 1480) = 1.51$ GPa.

\subsection{Plasma Formation and Optical Emissions}

Localized heating will generate plasma with the following characteristics:

\begin{align}
\text{Peak Temperature:} &\quad T_{\text{plasma}} > 100,000 \text{ K} \\
\text{Electron Density:} &\quad n_e \sim 10^{20} \text{ m}^{-3} \\
\text{Channel Diameter:} &\quad d_{\text{plasma}} \sim 1-5 \text{ mm} \\
\text{Lifetime:} &\quad \tau_{\text{plasma}} \sim 10 \text{ μs}
\end{align}

The plasma will emit characteristic spectral lines corresponding to ionized water components:
\begin{itemize}
\item Hydrogen: Balmer series (656.3 nm, 486.1 nm, 434.0 nm) \cite{griem1997principles}
\item Oxygen: Multiple ionic transitions (777.4 nm, 844.6 nm) \cite{ralchenko2008nist}
\item Continuum radiation from free-free transitions
\end{itemize}

\subsection{Čerenkov Radiation}

When charged particles exceed the phase velocity of light in the medium, Čerenkov radiation will be produced \cite{jelley1958cerenkov}:

\begin{equation}
\cos\theta_C = \frac{1}{n\beta}
\label{eq:cerenkov_angle}
\end{equation}

where $n = 1.33$ is the refractive index of water and $\beta = v/c$.

For $v = 2500$ m/s: $\beta = 8.33 \times 10^{-6}$, which is below the Čerenkov threshold. However, ionized particles in the plasma channel may achieve relativistic velocities, producing observable blue emission.

\subsection{Electromagnetic Signatures}

The rapid motion of ionized fluid will generate electromagnetic effects:

\begin{align}
\text{EMP Duration:} &\quad \sim 100 \text{ nanoseconds} \\
\text{Frequency Range:} &\quad 10^{6}-10^{9} \text{ Hz} \\
\text{Magnetic Field:} &\quad B \sim 10^{-3}-10^{-2} \text{ T}
\end{align}

The magnetic field can be estimated from the moving plasma current:

\begin{equation}
B = \frac{\mu_0 I}{2\pi r} = \frac{\mu_0 n_e e v A}{2\pi r}
\label{eq:magnetic_field}
\end{equation}

\section{Experimental Setup and Instrumentation}

\subsection{Track Design}

The experimental track consists of a 1-meter long, 10 cm diameter pressure vessel designed to withstand peak pressures of 10 GPa. The vessel incorporates:

\begin{itemize}
\item Ten 10 cm heating sections with independent temperature control
\item Pressure sensors at 5 cm intervals (20 sensors total)
\item Optical access ports for high-speed photography
\item Electrical connections for conductivity enhancement
\item Fluid circulation system for mixture homogenization
\end{itemize}

\subsection{Ballistic System}

The projectile delivery system utilizes a computer-controlled pneumatic rifle capable of:
\begin{itemize}
\item Adjustable muzzle velocity: 500-3000 m/s
\item Projectile mass range: 0.5-100 g
\item Firing rate: 3 projectiles per second
\item Timing precision: $\pm 10$ μs
\item Velocity measurement accuracy: $\pm 1\%$
\end{itemize}

\subsection{Diagnostic Equipment}

\subsubsection{High-Speed Imaging}
Multiple Phantom TMX 7510 cameras (1.75 million fps capability) positioned at:
\begin{itemize}
\item Entrance window: Projectile velocity measurement
\item Mid-track: Cavitation and plasma visualization
\item Exit window: Shock wave propagation imaging
\item Side ports: Perpendicular view of channel formation
\end{itemize}

\subsubsection{Pressure Measurements}
Twenty Kistler 603B1 pressure sensors with specifications:
\begin{itemize}
\item Response time: $< 1$ μs
\item Pressure range: 0-10 GPa
\item Accuracy: $\pm 0.5\%$
\item Sampling rate: 10 MHz
\end{itemize}

\subsubsection{Optical Spectroscopy}
Ocean Optics HR4000 spectrometer array for plasma characterization:
\begin{itemize}
\item Wavelength range: 200-1100 nm
\item Resolution: 0.02 nm
\item Integration time: 1 μs minimum
\item Temperature measurement via line broadening analysis
\end{itemize}

\subsubsection{Electromagnetic Detection}
RF spectrum analyzer and magnetic field sensors:
\begin{itemize}
\item Frequency range: 1 MHz - 1 GHz
\item Magnetic field sensitivity: 10⁻⁶ T
\item Temporal resolution: 10 ns
\item Bandwidth: 100 MHz
\end{itemize}

\section{Safety Considerations}

\subsection{Pressure Containment}

The high pressures generated (up to 5 GPa) require robust containment systems:
\begin{itemize}
\item Primary containment: Steel pressure vessel rated to 15 GPa
\item Secondary containment: Reinforced concrete bunker
\item Pressure relief systems with sub-millisecond response
\item Remote operation from protected control room
\end{itemize}

\subsection{Electrical Safety}

The 50 kV electrical system requires:
\begin{itemize}
\item Faraday cage construction around experimental area
\item Ground fault detection and automatic shutdown
\item Personnel isolation during operation
\item EMP shielding for sensitive electronics
\end{itemize}

\subsection{Chemical Hazards}

The fluid mixture contains flammable organic solvents:
\begin{itemize}
\item Explosion-proof electrical systems
\item Inert gas purging capabilities
\item Fire suppression systems
\item Vapor monitoring and ventilation
\end{itemize}

\section{Expected Results and Significance}

\subsection{Acoustic Measurements}

We anticipate recording the first controlled underwater sonic boom with:
\begin{itemize}
\item Peak pressures exceeding atmospheric sonic boom levels by 3-4 orders of magnitude
\item Complex reflection patterns from track boundaries
\item Acoustic signature unique to liquid media supersonic transitions
\item Validation of theoretical pressure calculations within $\pm 20\%$
\end{itemize}

\subsection{Plasma Characterization}

Expected plasma parameters:
\begin{itemize}
\item Electron density: $10^{19}-10^{21}$ m⁻³
\item Temperature: 50,000-200,000 K
\item Lifetime: 5-20 μs
\item Channel length: 50-100 mm
\end{itemize}

These measurements will provide the first comprehensive characterization of ballistically-induced plasma in aqueous media.

\subsection{Fluid Dynamics Insights}

The experiment will advance understanding of:
\begin{itemize}
\item Extreme cavitation dynamics at supersonic velocities
\item Multi-phase flow transitions under shock loading
\item Plasma-fluid interactions in dense media
\item Energy dissipation mechanisms in supersonic liquid flow
\end{itemize}

\section{Technological Applications}

\subsection{Underwater Propulsion}

Understanding supersonic underwater dynamics could enable:
\begin{itemize}
\item Supercavitating torpedo design optimization
\item Novel underwater vehicle propulsion concepts
\item Reduced drag through controlled cavitation
\item Enhanced underwater ballistic systems
\end{itemize}

\subsection{Materials Processing}

The extreme conditions generated offer applications in:
\begin{itemize}
\item Shock wave materials synthesis
\item Plasma-assisted chemical processing
\item High-pressure materials testing
\item Novel surface treatment techniques
\end{itemize}

\subsection{Energy Systems}

Plasma formation mechanisms could contribute to:
\begin{itemize}
\item Fusion ignition research
\item Plasma confinement studies
\item Energy storage in plasma states
\item Electromagnetic pulse generation
\end{itemize}

\section{Theoretical Implications}

\subsection{Fundamental Physics}

This experiment probes the intersection of:
\begin{itemize}
\item Classical fluid mechanics at extreme velocities
\item Plasma physics in non-equilibrium conditions
\item Thermodynamics of rapid phase transitions
\item Electromagnetic phenomena in moving media
\end{itemize}

\subsection{Computational Validation}

Results will provide crucial validation data for:
\begin{itemize}
\item Computational fluid dynamics codes at extreme conditions
\item Magneto-hydrodynamic simulation packages
\item Shock wave propagation models
\item Multi-phase flow solvers
\end{itemize}

\section{Conclusion}

We have presented a comprehensive framework for achieving Mach 1.7 supersonic velocities in aqueous media through a novel sequential projectile approach. The theoretical analysis demonstrates feasibility with existing technology, requiring approximately 3 kJ of input energy delivered through three precisely timed ballistic stages.

The expected phenomena—including underwater sonic booms with peak pressures of 5 GPa, plasma formation at temperatures exceeding 100,000 K, and complex electromagnetic signatures—will provide unprecedented insights into supersonic fluid dynamics in dense media.

This experiment represents the first practical approach to controlled underwater supersonic ballistics, opening new frontiers in:
\begin{enumerate}
\item Extreme fluid dynamics research
\item Plasma physics in liquid media
\item Shock wave engineering applications
\item Novel propulsion concept development
\end{enumerate}

The successful execution of this experiment will establish fundamental principles for supersonic motion in liquid media and provide validation data for next-generation computational models, while demonstrating the feasibility of achieving previously impossible velocities through innovative engineering approaches.

Future work will extend these principles to achieve higher Mach numbers, explore different fluid compositions, and investigate applications in underwater propulsion and materials processing. The techniques developed here lay the groundwork for a new field of supersonic liquid ballistics with broad implications for science and technology.

\section*{Acknowledgments}

The author acknowledges the theoretical foundations provided by classical fluid mechanics, plasma physics, and ballistics research. Special recognition is given to the pioneering work in cavitation dynamics, shock wave physics, and supersonic aerodynamics that made this theoretical framework possible.

\bibliographystyle{plain}
\begin{thebibliography}{99}

\bibitem{anderson2003fundamentals}
Anderson, J.D. (2003). \textit{Fundamentals of Aerodynamics}. McGraw-Hill Education.

\bibitem{bertin2013hypersonic}
Bertin, J.J., \& Cummings, R.M. (2013). \textit{Hypersonic Aerothermodynamics}. AIAA Education Series.

\bibitem{kinsler2000fundamentals}
Kinsler, L.E., Frey, A.R., Coppens, A.B., \& Sanders, J.V. (2000). \textit{Fundamentals of Acoustics}. John Wiley \& Sons.

\bibitem{pierce2019acoustics}
Pierce, A.D. (2019). \textit{Acoustics: An Introduction to Its Physical Principles and Applications}. Springer.

\bibitem{white2016fluid}
White, F.M. (2016). \textit{Fluid Mechanics}. McGraw-Hill Education.

\bibitem{brennen2014cavitation}
Brennen, C.E. (2014). \textit{Cavitation and Bubble Dynamics}. Cambridge University Press.

\bibitem{fridman2008plasma}
Fridman, A. (2008). \textit{Plasma Chemistry}. Cambridge University Press.

\bibitem{glass1974shock}
Glass, I.I., \& Sislian, J.P. (1974). \textit{Nonstationary Flows and Shock Waves}. Oxford University Press.

\bibitem{hirt1974volume}
Hirt, C.W., \& Nichols, B.D. (1974). Volume of fluid (VOF) method for the dynamics of free boundaries. \textit{Journal of Computational Physics}, 39(1), 201-225.

\bibitem{shi2010numerical}
Shi, H.H., \& Takami, T. (2010). Numerical simulation of supercavitating flow around high-speed underwater projectiles. \textit{Computers \& Fluids}, 39(8), 1448-1456.

\bibitem{plesset1977bubble}
Plesset, M.S., \& Prosperetti, A. (1977). Bubble dynamics and cavitation. \textit{Annual Review of Fluid Mechanics}, 9(1), 145-185.

\bibitem{bruggeman2002plasma}
Bruggeman, P., \& Leys, C. (2009). Non-thermal plasmas in and in contact with liquids. \textit{Journal of Physics D: Applied Physics}, 42(5), 053001.

\bibitem{cengel2019thermodynamics}
Çengel, Y.A., \& Boles, M.A. (2019). \textit{Thermodynamics: An Engineering Approach}. McGraw-Hill Education.

\bibitem{chen2016introduction}
Chen, F.F. (2016). \textit{Introduction to Plasma Physics and Controlled Fusion}. Springer.

\bibitem{sedov1993similarity}
Sedov, L.I. (1993). \textit{Similarity and Dimensional Methods in Mechanics}. CRC Press.

\bibitem{schlichting2016boundary}
Schlichting, H., \& Gersten, K. (2016). \textit{Boundary-Layer Theory}. Springer.

\bibitem{pope2000turbulent}
Pope, S.B. (2000). \textit{Turbulent Flows}. Cambridge University Press.

\bibitem{lieberman2005principles}
Lieberman, M.A., \& Lichtenberg, A.J. (2005). \textit{Principles of Plasma Discharges and Materials Processing}. John Wiley \& Sons.

\bibitem{landau1987fluid}
Landau, L.D., \& Lifshitz, E.M. (1987). \textit{Fluid Mechanics}. Pergamon Press.

\bibitem{blake1987cavitation}
Blake, F.G. (1949). The onset of cavitation in liquids. \textit{Journal of the Acoustical Society of America}, 21(6), 616-619.

\bibitem{bird2006transport}
Bird, R.B., Stewart, W.E., \& Lightfoot, E.N. (2006). \textit{Transport Phenomena}. John Wiley \& Sons.

\bibitem{zel1967physics}
Zel'dovich, Y.B., \& Raizer, Y.P. (1967). \textit{Physics of Shock Waves and High-Temperature Hydrodynamic Phenomena}. Academic Press.

\bibitem{griem1997principles}
Griem, H.R. (1997). \textit{Principles of Plasma Spectroscopy}. Cambridge University Press.

\bibitem{ralchenko2008nist}
Ralchenko, Y., Kramida, A.E., Reader, J., \& NIST ASD Team. (2008). NIST Atomic Spectra Database. National Institute of Standards and Technology.

\bibitem{jelley1958cerenkov}
Jelley, J.V. (1958). \textit{Čerenkov Radiation and Its Applications}. Pergamon Press.

\end{thebibliography}

\end{document}
